\documentclass[11pt,]{article}
\usepackage[left=1in,top=1in,right=1in,bottom=1in]{geometry}
\newcommand*{\authorfont}{\fontfamily{phv}\selectfont}
\usepackage[]{mathpazo}


  \usepackage[T1]{fontenc}
  \usepackage[utf8]{inputenc}



\usepackage{abstract}
\renewcommand{\abstractname}{}    % clear the title
\renewcommand{\absnamepos}{empty} % originally center

\renewenvironment{abstract}
 {{%
    \setlength{\leftmargin}{0mm}
    \setlength{\rightmargin}{\leftmargin}%
  }%
  \relax}
 {\endlist}

\makeatletter
\def\@maketitle{%
  \newpage
%  \null
%  \vskip 2em%
%  \begin{center}%
  \let \footnote \thanks
    {\fontsize{18}{20}\selectfont\raggedright  \setlength{\parindent}{0pt} \@title \par}%
}
%\fi
\makeatother




\setcounter{secnumdepth}{0}



\title{Implementing the Heston Option Pricing Model in Object-Oriented Cython  }



\author{\Large Brandon C. Hardin\vspace{0.05in} \newline\normalsize\emph{Utah State University}  }


\date{}

\usepackage{titlesec}

\titleformat*{\section}{\normalsize\bfseries}
\titleformat*{\subsection}{\normalsize\itshape}
\titleformat*{\subsubsection}{\normalsize\itshape}
\titleformat*{\paragraph}{\normalsize\itshape}
\titleformat*{\subparagraph}{\normalsize\itshape}


\usepackage{natbib}
\bibliographystyle{apsr}



\newtheorem{hypothesis}{Hypothesis}
\usepackage{setspace}

\makeatletter
\@ifpackageloaded{hyperref}{}{%
\ifxetex
  \usepackage[setpagesize=false, % page size defined by xetex
              unicode=false, % unicode breaks when used with xetex
              xetex]{hyperref}
\else
  \usepackage[unicode=true]{hyperref}
\fi
}
\@ifpackageloaded{color}{
    \PassOptionsToPackage{usenames,dvipsnames}{color}
}{%
    \usepackage[usenames,dvipsnames]{color}
}
\makeatother
\hypersetup{breaklinks=true,
            bookmarks=true,
            pdfauthor={Brandon C. Hardin (Utah State University)},
             pdfkeywords = {},  
            pdftitle={Implementing the Heston Option Pricing Model in Object-Oriented Cython},
            colorlinks=true,
            citecolor=blue,
            urlcolor=blue,
            linkcolor=magenta,
            pdfborder={0 0 0}}
\urlstyle{same}  % don't use monospace font for urls



\begin{document}
	
% \pagenumbering{arabic}% resets `page` counter to 1 
%
% \maketitle

{% \usefont{T1}{pnc}{m}{n}
\setlength{\parindent}{0pt}
\thispagestyle{plain}
{\fontsize{18}{20}\selectfont\raggedright 
\maketitle  % title \par  

}

{
   \vskip 13.5pt\relax \normalsize\fontsize{11}{12} 
\textbf{\authorfont Brandon C. Hardin} \hskip 15pt \emph{\small Utah State University}   

}

}







\begin{abstract}

    \hbox{\vrule height .2pt width 39.14pc}

    \vskip 8.5pt % \small 

\noindent The 1973 Black-Scholes model, a revolutionary option pricing formula
whose price is `relatively close to observed prices, makes an asumption
that the volatility is constant and thus, deterministic. This causes
some inefficiencies and patterns in the pricing of options due to the
model providing evidence of the 'volitility smile' of the volatility.
Many scholars have suggested that the volatility should be modelled by a
stochastic process and the (1993) Heston Model is one of many proposed
solutions to remedy this problem. The Heston Model allows for the
`smile' by defining the volatitliy as a stochastic process. This thesis
considers a solution to this problem by utilizing Heston's stochastic
volitility model in conjunction with Euler's discretization scheme in a
simple Monte Carlo engine. The application of this model has been
implemented in object-oriented Cython, for it provides the simplicity of
Python, all the while, providing C-like performance.


    \hbox{\vrule height .2pt width 39.14pc}


\end{abstract}


\vskip 6.5pt

\noindent \doublespacing \subsection{1.Introduction}\label{introduction}

This thesis shall simulate the Heston Model by the use of Cython and
thus, the reasoning behind the chosen model must be identified. The
primary factor for the decision to implement the Heston Model was how it
determines the evolution of volatility of the underlying asset. With the
aid of continuous time diffusion models for volatility, the Heston Model
derives its option pricing from a random process. Although the
Black-Scholes model is widely supported, it is subject to
inefficiencies, error and incorrectly pricing securities due to the
assumption of constant volatility. If the Black-Scholes assumption were
correct than, when viwing strike prices, the implied volatilities of
same type options would be constant. This is not the case as patterns of
volatilities varying by strike can be seen forming a smile curve or
``volatility smile''. Stochastic volatility models were formulated in
order to sove this problem. These models incorporate the verifiable
observations, for which, the volatility of the model is a random
process. In turn, volatility itself is made into stochastic process.
Developed by Steven Heston, the widely used Heston (1993) model not only
took time-dependent volatility into account, it also presents a
stochastic process element. The Heston Model provides correlated shocks
between asset returns and volatility. This assumption provides insight
into the reasoning behind return skewness and strike-price biases in the
Black-Scholes model. In addition, the Heston provides for the actuality
of a semi-analytical resolution for European options. In this thesis, we
shall institute a simulation of one of the most widely used stochastic
volatility models, the Heston Model's volatility stochastic process. The
implementation is inclusive of random-number generation in a Monte Carlo
engine. Monte Carlo simulation is a vital technique used in option
pricing as it not only provides an improvement in the efficiency of a
simulation, but it does so by sampling a values randomly from all
possible outcomes from the input probability distributions. The Monte
Carlo simulation does this iteration as many times as specified and the
result is a probability distribution of all possible outcomes. The Monte
Carlo simulation implementation is quantified in Cython within the
Python software . Python is a high-level programming language that is
used in a variety of technical areas including finance. Although, Python
is widely used for option pricing theory, the execution of the
forementioned within a Cython environment is realtively new. In the next
portion of this thesis, the Heston (1993) model shall be presented. \#\#
2.Black-Scholes The most recognized widely used continuous time model is
the Black-Scholes. With its simplistic nature and requirement of only
five inputs: strik price, asset price, expiry, risk-free rate, and
volatility, the Black-Scholes makes an assumption of an underlying
asset,S, where it follows a geometric Brownian motion and we assume the
drift and volatility is constant. The plain Black-Scholes model process
is as follows: \[
dS_{t} = \mu S_{t} dt + \sigma S_{t} dW_t,                          (2.1)
\] where both\(\mu\), the drift and \(\sigma\), the volatility, are
under the assumption of being constant. As the result of asset price
changes are lognormal-distributions, which means that the values are
positive and they create a right skewed curve, which is a disadvantage
of the model. In addition, the Black-Scholes model has demonstrated an
issue with being consistant with the market as far as implied volatility
is concerned. Allow \(C_BS(S,K,T,r,I) = C_market)\) to denote the
Black-Scholes price for a European call option with T time to expiry,
strike price K, S is the value of the underlying asset, r is the
risk-free rate and implied vol is the value of I, while making note that
vol, which has a drastic effect on price is not visible. When
calculating implied volatilites from market data, the same sigma should
be detectable for all options on the same underlying asset, but this is
not what is observed. The implied volatilities derived from market data
are not constant as they vary with the strike price and time to expiry
even when associated with the same underlying asset, which entails the
formation of a skew or ``volatility smile.'' The implied volatilities
also vary over the course of time in a stochastic fashion. This directly
repudiates the assumptions of constant volatility that Black-Scholes
states. A stochastic volatility model can remedy the Black-Scholes of
this contradiction.

\subsection{3.The Heston Model}\label{the-heston-model}

The evolution of the volatility of an underlying asset provides the
reasoning behind the creation of the Heston Model. When there is an
correlation between the asset price and volatility, it produces a
closed-form solution and allows the model to make the addition of
stochastic interest rates. The Heston (1993) model is based upon the
following stochastic differential equations, which depicts the stock
price and variance process diffusions as: \[
dS_{t} = \mu S_{t} dt + \sqrt{v_{t}} S_{t} dz_{1,t}                 (3.1)
\] \[ 
d\sqrt{v_{t}} = -\beta \sqrt{v_{t}} dt + \delta dz_{2,t}            (3.2)
\] Equation (2.1) assumes the underlying asset price follows the
diffusion process at time t where where \(\mu\) is the drift parameter,
and \(z_{1,t}\) is a standard Wiener process (i.e.~random walk). In
equation (2.2) the volatility \(\sqrt{v_{t}}\) itself follows a
diffusion process where \(z_{2,t}\) is a Wiener process, and \(\rho\)
defines the correlation between \(z_{1,t}\) and \(z{2,t}\).

Ito's lemma shows the variance process can be written as follows as the
volatility follows an Orstein-Uhlenbeck (mean-reverting) process: \[
dv_{t} = \kappa [\theta - v_{t}] dt + \sigma \sqrt{v_{t}} d z_{2,t} (3.3)
\] where \(\theta\) is the long-run mean of the variance, \(\kappa\) is
a mean reversion parameter and \(\sigma\) is the volatility of
volatility. Utilizing this layout, the Heston (1993) provides the price
of the call option as: \[
Call(S,v,t) = S_{t} P_{1} - K P(t,T) P_{2}                          (3.4)
\] where \(S_{t}\) is the spot price of the asset, \(K\) is the strike
price of the option, \(P(t,T)\) is a discount factor from time \(t\) to
time \(T\). If we assume a constant rate of interest \(r\), we can write
this as: \[
P(t,T) = \exp{-r \times (T - t)}                                    (3.5)
\] Where \(P_{1}\) and \(P_{2}\) are the probabilities that the call
option will expire in-the-money, subject to the log stock price
\(x_{t} = ln[S_{t}]\), and on the volatility \(v_{t}\), each at time
\(t\). Under the risk-neutral measure, these versions are given by: \[
\begin{aligned} dx_{t} &= \left[r -\frac{1}{2}v_{t} \right] dt + \sqrt{v_{t}} dz_{1,t}^{\ast} \ dv_{t} &= \kappa \left[\theta - v_{t} \right] dt + \sigma \sqrt{v_{t}} dz_{2,t}^{\ast} \end{aligned}                                          (3.6)
\] We have to descretize these SDE's in order to simulate them. Although
there are no closed-form solutions for equaitons (2.6), we can still use
the Euler disretization technique. But the draws variance equation
produces will be negative values of volatility. While there are many
ways to rectify this, we shall work directly with the natural logarithm
of the varience. By utilizing It's lemma once more, we get the dynamics
of \(\ln{(v_{t})}\) as follows: \[ 
d \ln{(v_{t})} = \frac{1}{v_{t}} ( \kappa (\theta - v_{t}) - \frac{1}{2} \sigma^{2}) dt + \sigma\frac{1}{\sqrt{v_{t}}} dz_{2,t}^{\ast}       (3.7)
\] Now we can execute the Euler discretization technique and we now have
the following discrete time solutions:
\[ \begin{aligned} \ln{(S_{t + \Delta t})} &= \ln{(S_{t})} + \left(r - \frac{1}{2} v_{t}\right) \Delta t + \sqrt{v_{t}}
\sqrt{\Delta t} \varepsilon_{S,t+1}\ \ln{(v_{t + \Delta t})} &= \ln{(v_{t})} + \frac{1}{v_{t}} \left(\kappa (\theta -
v_{t}) - \frac{1}{2} \sigma^{2}\right) \Delta t + \sigma \frac{1}{\sqrt{v_{t}}} \sqrt{\Delta t} \varepsilon_{v,t+1}
\end{aligned}                                                       (3.8)
\] Shocks to the volatility, \(\varepsilon_{v,t+1}\) are correlated with
the shocks to the stock price process, \(\varepsilon_{S,t+1}\). This is
denoted by \(\rho = Corr(\varepsilon_{v,t+1}, \varepsilon_{S,t+1})\) and
the relationship between shocks can be written: \[
 \varepsilon_{v,t+1} = \rho \varepsilon_{S,t+1} + \sqrt{1 - \rho^{2}} \varepsilon_{t+1}                 (3.9)
\] where the \(\varepsilon_{t+1}\) are iid Standard Normal variables
that each have zero correlation with \(\varepsilon_{S,t+1}\).

\subsection{4.Monte Carlo Simulation}\label{monte-carlo-simulation}

\subsection{5.Python Programming
Language}\label{python-programming-language}

From the Python website, you can find Python's executive
summary(\url{https://www.python.org/doc/essays/blurb/}): Python is an
interpreted, object-oriented, high-level programming language with
dynamic semantics. Its high-level built in data structures, combined
with dynamic typing and dynamic binding, make it very attractive for
Rapid Application Development, as well as for use as a scripting or glue
language to connect existing components together. Pythons simple, easy
to learn syntax emphasizes readability and therefore reduces the cost of
program maintenance. Python supports modules and packages, which
encourages program modularity and code reuse. The Python interpreter and
the extensive standard library are available in source or binary form
without charge for all major platforms, and can be freely distributed.
\#\# 6.Impementing Option Pricing Models in Python

\subsection{7.Python}\label{python}

\subsection{8.Cython}\label{cython}

\subsection{9.Object-Oriented
Programming}\label{object-oriented-programming}

\subsection{10.Design Patterns in Computational
Finance}\label{design-patterns-in-computational-finance}

\subsection{10.The Heston Model in
Cython}\label{the-heston-model-in-cython}

\subsection{12.Numerical Examples}\label{numerical-examples}

\subsection{13.Section with Math}\label{section-with-math}

This is some latex: \(\mu\) and \(\sigma\)

This is an equation:

\[
dS_{t} = \mu S_{t} dt + \sigma S_{t} dZ_{t}
\]

\newpage
\singlespacing 
\bibliography{master.bib}

\end{document}
